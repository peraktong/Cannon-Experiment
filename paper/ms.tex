\documentclass[11pt, oneside]{article}   	% use "amsart" instead of "article" for AMSLaTeX format
\usepackage{geometry}                		% See geometry.pdf to learn the layout options. There are lots.

\geometry{letterpaper}                   		% ... or a4paper or a5paper or ... 
%\geometry{landscape}                		% Activate for rotated page geometry
%\usepackage[parfill]{parskip}    		% Activate to begin paragraphs with an empty line rather than an indent
\usepackage{graphicx}				% Use pdf, png, jpg, or eps§ with pdflatex; use eps in DVI mode
								% TeX will automatically convert eps --> pdf in pdflatex		
\usepackage{amssymb}
\usepackage{amsmath, amsthm, amssymb, amsfonts}

\usepackage{dcolumn}% Align table columns on decimal point
\usepackage{bm}% bold math

\usepackage[T1]{fontenc}
\usepackage[utf8]{inputenc}
\usepackage{authblk}


%SetFonts

%SetFonts


\title{CannonRV: Measuring radial velocities with a data-driven spectral model}
\author[1]{Jason Cao\thanks{jc6933@nyu.edu}}
\author[2]{David W. hogg\thanks{david.hogg@nyu.edu}}
\author[3]{Melissa Ness\thanks{ness@mpia-hd.mpg.de}}

\affil[1]{Department of Physics,  New York University}
\affil[2]{NYU Physics - Center for Cosmology and Particle Physics
NYU Center for Data Science
Max-Planck-Institut fuer Astronomie }
\affil[3]{
Max-Planck-Institut Max-Planck-Institut fuer Astronomie 17, D-69117 Heidelberg, Germany
}

\renewcommand\Authands{ and }

  
\date{\today}						% Activate to display a given date or no date



\begin{document}
\maketitle


\section{\label{sec:level1}Introduction}


Coming soon...
 
 %% second paragraph
 
 \section{\label{sec:level1}Assumption and method}
 \subsection{\label{sec:level2}Optimize the spectrum}

 
Suppose we have a trained model. The training set contains 548 stellar and each stellar spectrum has 8575 pixels. Every pixel is related to a specific wave length. It belongs to APOGEE data release 10 and has a good quality. These stellar are the very same objects as used by the APOGEE survey for the calibration of DR10. Each of them has at least nine stellar labels. But in the training step, we only use three of them $T_{eff}$, $log$ g  and $\left[Fe/H\right]$. They span the range of $3500K <T_{eff}< 5300K$, $0 <log$ $g< 5$ and $-2.5 <\left[Fe/H\right]< 0.45$.
]

The model we adopt

\begin{equation}
y_{jn} = v(l_n) \cdot \theta_{j} + e_{jn}
\end{equation}

                   
 
 Where \(y_{jn}\) is the spectrum data for star n at wavelength pixel j. 
 \(v(l_n)\) is the vectorizing function. 
The input \(l_n\) is the label list of length K for star n and the  output 
\(v(l_n)\) is a vector of length D (D is bigger than K).
\( \theta_{j}\) is a vector of length D of parameters which controlling the model at wavelength pixel j.
\( e_{jn}\) is a noise draw or residual at pixel j for star n.


\(l_n^{inf}\) and \( \theta_{j}\) are available after the training step.
Infer spectrum by using:

\begin{equation}
y_{jn}^{inf} = v(l_n^{inf}) \cdot \theta_{j} 
\end{equation}

 
 This is how to obtain the synthesized spectrum by using the Cannon 2. We also call it inferred flux(spectrum).

The Cannon 2 can predict the spectrum pretty well. But we can still improve it.
\bigskip

From the inferred spectrum \(y_{jn}^{inf}\), we optimize it and generate a better synthesized spectrum, which we call \(y_{jn}^{opt}\).
\bigskip

The method we use is to represent the spectrum at pixel j by using the inferred spectrum at pixel j-1,pixel j and pixel j+1.(If the pixel is out of range, set the flux as 1). The data we use is the same as the training step.
\bigskip

The new spectrum, which is the optimized spectrum, is represented by:

\begin{equation}
y_{j,n}^{opt}=a\cdot y_{j-1,n}^{inf}+b\cdot y_{j,n}^{inf}+c\cdot y_{j+1,n}^{inf}
\end{equation}
\bigskip

We use the least chi-squared method. So our objective function is the chi-squared between the optimized spectrum and the spectrum data. The objective function for star n is:
\bigskip

\begin{equation}
F^{obj}_n (a,b,c)= \sum_{j=1}^{j=J} {(y_{jn} - y_{jn}^{opt})^2 \over{\sigma_{jn}^2}}
\end{equation}


$\sigma_{jn}^2$ is the data uncertainty for star n at pixel j. The number of stellar and pixel each star are N and J. Here N and J are the number of the stellar and the number of pixel each star.  Write the objective function in the form of matrix can make it simpler. There will be N objective functions and N sets of a, b and c. So we introduce several matrix: $Y_n$, $A_n$, $C_n$ and $X_n$, which are defined as followed:

% insert matrix


\[
 Y_n
=
\begin{bmatrix}
    y_{1,n} \\
    y_{2,n} \\
    \vdots  \\
    y_{J,n}  \\
    
\end{bmatrix}
\]

% A

\[
A_n
=
\begin{bmatrix}
    
    1 & y_{1,n} ^{inf} & y_{2,n} ^{inf}\\
    y_{1,n}^{inf}&y_{2,n}^{inf}& y_{3,n}^{inf}  \\
    \vdots & \vdots & \vdots \\
    y_{J-1,n}^{inf}&y_{J,n}^{inf}& 1  \\

\end{bmatrix}
\]



% sigma-C


\[
C_n
=
\begin{bmatrix}
    
    \sigma_{1,n}^2&0&\dots&0\\
    0& \sigma_{2,n}^2&\dots&0 \\
    \vdots & \vdots & \vdots \\
    0&\dots&0&\sigma_{J,n}^2  \\

\end{bmatrix}
\]
 
\bigskip

\begin{center}
\(X_n = 
 \begin{bmatrix}
    
    a_n\\
    b_n \\
    c_n \\

\end{bmatrix}
\)
\end{center}



Now the objective function can be described as:

\begin{equation}
F^{obj}_n (a_n,b_n,c_n)= (Y_n-A_n X_n)^TC^{-1}(Y_n-A_n X_n)
\end{equation}


And the Jacobian Matrix of the objective function with respect to parameters is

\begin{equation}
J = {\partial F^{obj}_n\over \partial X_n } = 2\cdot [(Y_n-A_nX_n)^TC^{-1}]\cdot (-A_n)
\end{equation}



The dimension of the Jacobian Matrix is 1*3. To minimize the objective function, let the Jacobian Matrix be 0 and we have:

\begin{equation}
X_n = [\sum_{n=1}^{n=N}A_n^TC_n^{-1}A]^{-1} \cdot [\sum_{n=1}^{n=N}A^TC_n^{-1}Y_n]
\end{equation}


By using (7) we can calculate $a_n$, $b_n$ and $c_n$, which are parameters a, b and c for star n. 
b should be much bigger than both a and c. And a+b+c should be equal to 1. 


Calculate the optimized spectrum \( y_{j,n}^{opt}\) by using (3). The Experiment 1 shows that the optimization works.
\bigskip




%It works!


\subsection{\label{sec:level2}Spectrum calibration}
\bigskip
After the optimization, we obtain many sets of a, b and c. In our theory b should be close to 1 and a, c is much smaller than b. But the fact is that sometimes b is much smaller than and c. This is because of the spectral broadening in the measurement. The equipment is not perfect, so the resolution of the spectrum will become lower due to scattering, dispersion or other reasons. Then the spectrum will broaden and the real spectrum at pixel j is not what we obtain, it should be the spectrum at nearby pixels, which include pixel j-1 and j+1. Thus b will be smaller than a+c, which we will discuss in the following experiment 3.
\bigskip

Sometimes a or c will be much bigger than the other two parameters. This is also because of the measurement. Even we ignore the spectral broadening, the measured spectrum may move one pixel left or one pixel right from the real spectrum. If the measured spectrum move one pixel left, then a should be close to 1 and it's similar for c. The measured spectrum moves left or right is because of the error in the measured velocity, so there will be error in measuring the Doppler effect, which leads to a spectrum moving. We will show this in experiment 3.
\bigskip



\subsection{\label{sec:level2}Optimize Teff , log g and [Fe/H] simultaneously with a, b, c}

Working on it...



%% Third paragraph
 
 \section{\label{sec:level1}Experiment}
 \subsection{\label{sec:level2}Experiment 1}
 
Choose four stellar, which are not in the training set. Optimize the spectrum of these four stars and obtain a, b and c for each stars.
 
The labels of the four stars are:
\begin{center}
 
 Star A  $T_{eff} =4750$  log g =3.0 $\left[Fe/h\right] =0.15$
 
 Star B  $T_{eff} =4849$  log g =2.2 $\left[Fe/h\right] =-1.0$
 
 Star C  $T_{eff} =3614$  log g =0.4 $\left[Fe/h\right] =-0.68$
 
 Star D  $T_{eff} =5003$  log g =2.8 $\left[Fe/h\right] =-0.71$
 
 \end{center}
 
 
\begin{table}[ht]
\caption{Fitting results for the four stars}
\centering
\begin{tabular}{c c c c c}
\hline\hline
Star Name & a & b & c & a+b+c \\ [0.5ex] % inserts table %heading
\hline
Star A & 0.121 & 0.824 & 0.052 & 0.998 \\
Star B & 0.176 & 0.541 &  0.278 & 0.995 \\
Star C & 0.003 & 1.004 &  -0.007 & 1.000 \\
Star D & -0.009  & 1.029 & -0.020 & 1.000 \\ [1ex]
\hline
\end{tabular}
\label{table:nonlin}
\end{table}
 

 
 
 Then plot the spectrum
 
 The black line "Observed" is the observed spectrum from the data  \(y_{jn}\).
 The green line "inferred" is \( y_{j,n}^{inf}\) , which is generated by the Cannon 2.
 The red line "optimized" is \( y_{j,n}^{opt}\), which is generated by (3)
 
\bigskip


 
\includegraphics[width=170mm]{figure_1.png}
 \centerline{Fig.1 The optimized spectrum and the data spectrum}
\bigskip



\includegraphics[width=170mm]{figure_2.png}
 \centerline{Fig.2 The inferred spectrum and the data spectrum}
 
 

 
Finally compare the chi-square for these four stars of the two methods:
The total chi-squared of the four stars becomes 2\% smaller, which means our method works.
  
 \subsection{\label{sec:level2}Experiment 2}
 Plot the parameters a, b and c vs the number of the star. 
 \bigskip
 
In experiment 1, we use only four stars to obtain parameters a, b and c and the result is pretty good.  What if we optimize more stars? We randomly choose about 1\% of the APOGEE DR10 and make a histogram plot. The number of stars is 614. Now parameters a, b and c are obtained from independent stars, which means the objective function only contain the chi-squared of one star and there will be 614 objective functions. And there will be 614 sets of a, b and c. 


We plot a, b, c against the mean fiber number, which is the average fiber number of the spectrum we optimize. In APOGEE DR 10, the fiber number is between 1 and 640. The parameters we use are from figure 3

\includegraphics[width=170mm]{figure_3.png}
\bigskip

\centerline{Fig.3 a,b,c vs Mean Fiber Number}
\bigskip

This figure gives us information about the relation between parameters a, b, c and fiber number, which can be used to establish a data-driven spectrum calibration model.
\bigskip

We also plot delta-chi-squared vs Mean Fiber number, where delta-chi-squared is the difference between the chi-squared of the spectrum before our optimization and after. It should be bigger than 0.
\bigskip

\includegraphics[width=170mm]{figure_4.png}
\bigskip

\centerline{Fig.4 Delta-chi-squared vs Mean Fiber Number}
\bigskip

\subsection{\label{sec:level2}Experiment 3}

\bigskip
Now choose some examples from the 614 stars and make some plots.


First, plot the four stars with biggest delta-chi-squared and smallest delta-chi-squared:

\includegraphics[width=170mm]{figure_5.png}
\bigskip

\centerline{Fig.5 Spectrum of the four stars with biggest delta-chi-squared. }

\centerline{The black line is the observed spectrum. The green line is the inferred spectrum and the red line is the optimized spectrum.}
\bigskip


\includegraphics[width=170mm]{figure_6.png}
\bigskip

\centerline{Fig.6 Spectrum of the four stars with smallest delta-chi-squared. }

\centerline{The black line is the observed spectrum. The green line is the inferred spectrum and the red line is the optimized spectrum.}
\bigskip

From figure 6 and 7, we find that the four stars with smallest delta-chi-squared have b closer to 1, but the four stars with biggest delta-chi-squared don't.

Also, we want to look into the spectral broadening.
Plot four stars with b closest to 1:
\bigskip

\includegraphics[width=170mm]{figure_7.png}
\bigskip

\centerline{Fig.7 Spectrum of the four stars with b closest to 1. }

\centerline{The black line is the observed spectrum. The green line is the inferred spectrum and the red line is the optimized spectrum.}
\bigskip

From the histogram of a, b and c, we also find that sometimes b becomes much smaller than a+c, which means the spectrum is broadening due to the measurement. The resolution of the spectrum becomes lower and this is what we want to calibrate.
\bigskip
Here is the plot of the four stars with biggest a+c-b:
\bigskip

\includegraphics[width=170mm]{figure_8.png}
\bigskip

\centerline{Fig.8 Spectrum of the four stars with biggest a+c-b. }

\centerline{The black line is the observed spectrum. The green line is the inferred spectrum and the red line is the optimized spectrum.}
\bigskip


\bigskip

We found a and c are close to 1, which is in accordance with our theory. 
Also, the optimized spectrum broaden when a+c is much bigger than b, which means the measured spectrum broaden due to the error in the measurement. This is what we need to calibrate.
\bigskip

To look deeper into the results, we also choose the four stars with biggest $a-c$ and biggest $c-a$. Then make another two plots:
\bigskip

\includegraphics[width=170mm]{figure_9.png}
\bigskip

\centerline{Fig.9 Spectrum of the four stars with biggest $a-c$.}

\centerline{ The black line is the observed spectrum. The green line is the inferred spectrum and the red line is the optimized spectrum.}
\bigskip



\bigskip

\includegraphics[width=170mm]{figure_10.png}
\bigskip

\centerline{Fig.10 Spectrum of the four stars with biggest $c-a$.}

\centerline{ The black line is the observed spectrum. The green line is the inferred spectrum and the red line is the optimized spectrum.}
\bigskip

For spectrum with $a>c$, the optimized spectrum moves right a little bit, which means the real spectrum should move left a little. It's similar for $c>a$

\subsection{\label{sec:level2}Experiment 4}

\bigskip
We also plot $Teff$ against log g. The color bar shows the mean inverse variance of these stars.

\includegraphics[width=170mm]{figure_11.png}
\bigskip

\centerline{Fig.11 $Teff$ vs $log g$}
\bigskip

  
  
 %% Fourth paragraph

 \section{\label{sec:level1}Discussion}
 
 Coming soon.
 
  %% Fifth paragraph

 \section{\label{sec:level1}Appendix}
 
 
 Now we exclude stars with undetermined labels and re-plot the histogram of a, b and c.


\includegraphics[width=170mm]{d_figure_2.png}
\bigskip

\centerline{Fig.2 Histogram of a b c for apstar data (What we use now)}
\bigskip



\end{document}  